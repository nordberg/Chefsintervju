%% ABSTRACT

Denna rapport är resultatet av en intervju som genomfördes av tre studenter med en chef på Ericsson. Rapporten beskriver hur en chefsroll för ett teknikorienterat modernt företag kan se ut; vilka svårigheter som kan uppstå och vilka egenskaper en bra ledare bör ha. Kursens material ställs i proportion mot intervjusvaren för att se vilken betydelse innehållet har för det kommande arbetslivet.

Den slutsats som dras i och med denna rapport är dels att kravet på detaljkunskap avtagit för moderna ledare och dels att synen på ledare förändrats. Kravet på detaljkunskap har ersatts med en utökad önskan om att kunna motivera sina anställda, det är numera viktigare att få anställda att förstå sammanhanget och att chefen pekar ut målet än att tydligt avgöra vad som ska göras och hur. Synen som förändrats på ledare hör ihop med detta och innebär nu inte längre att en ledare ska bestämma riktning och hur företaget når dit utan överlåta det till mer insatta personer. Den moderna ledaren måste dock fortfarande hålla sig på en viss kunskapsnivå då det finns en kulturskillnad gentemot andra länder.

Vidare konstateras det att transformationen till det agila arbetssättet påverkat ledarrollen positivt. Dock kan chefsrollen ibland kännas redundant för personen. Detta är självklart något som är högst individuellt och ska ej ses som representativt för ledare i allmänhet.