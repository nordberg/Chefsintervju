%% Intervju
\section{Intervju}
När det gäller egenskaper en god ledare ska ha nämns \textit{förmågan att få folk bakom sig} flera gånger av John McCarthy. Det finns en signifikant skillnad mellan denna egenskap och \textit{förmågan att delegera arbete} som kanske oftare förknippas med ledarrollen. I John McCarthys formulering ingår någon slags motivation. Det handlar inte bara om att berätta för sina anställda vad de ska göra utan också varför de ska göra det.

Under en föreläsning i kursen togs begreppen intern och extern motivation upp\citep{motivation}. Den externa motivationen är belöningar och straff för det man åstadkommer, det kan innebära att man utför arbetet för att nå en befordran eller för att undvika att få sparken. En intern motivation handlar istället om att man gör något av egen vilja. Det är inte externa faktorer som gör att personen i fråga vill utföra uppgiften utan för att den känns relevant och viktig.

\begin{quote}
 Sen kan ju jag som chef säga riktningen och säga ``Vi ska ditåt, vi ska uppför det där berget.''. I min värld säger de ``Bra, stå åt sidan så fixar vi det.'' istället för ``Jaha, hur ska vi göra det?''.
\end{quote}

John McCarthy berättar om hur han ser på det moderna ledarskapet. Han berättar om hur en ledare inte nödvändigtvis ska veta hur något göras, det bör lämnas till experter inom området som jobbar med exempelvis koden varje dag. Istället är det upp till ledaren att  visa vilken riktning man ska ta och varför.

Detta är något som orsakade orolighet när Ericsson valde att gå över till ett agilt arbetssätt för ungefär sex år sedan. John McCarthy beskriver hur han kunde känna att chefsrollen blev något redundant. Ibland kunde frustration uppstå när chefen inte längre stod för lösningen. Vidare konstaterar han dock att när han tidigare var tvungen att visa vägen blev det väldigt ofta fel.

Vidare berättar John McCarthy att just synen på hur insatt i detaljfrågor en chef bör vara kan leda till kulturkrockar. På många platser runt om i världen är fortfarande tankesättet att det är upp till chefen att ha kunskap. John McCarthy anser sig dock hittat en lösning i att han är nyfiken på vad som sker. Det finns en intern motivation att lära sig om vissa delar av projektet vilket leder till att han kan svara på vissa frågor men går man in i detalj är det bättre att personen pratar med de anställda som jobbar med utvecklingen dagligen.

Generellt ger John McCarthy ett ödmjukt intryck. Han är noga med att framhäva att kunskapen ska finnas hos de anställda som utför arbetet. När han får frågan om hur han ser sig själv som ledare använder han ord som transperant och energisk. Detta är också något som han tror att hans anställda ser hos honom. Dock tror han att hans anställda ser på honom som en ledare med höga krav. Ambitionen är att ta in folk som är bättre än han själv.