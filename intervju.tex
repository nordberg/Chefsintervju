%% Intervju
\section{Intervju}
När det gäller egenskaper en god ledare ska ha nämns \textit{förmågan att få folk bakom sig} flera gånger av John McCarthy. Det finns en signifikant skillnad mellan denna egenskap och \textit{förmågan att delegera arbete} som kanske oftare förknippas med ledarrollen. I John McCarthys formulering ingår någon slags motivation. Det handlar inte bara om att berätta för sina anställda vad de ska göra utan också varför de ska göra det.

Under en föreläsning i kursen togs begreppen intern och extern motivation upp\citep{motivation}. Den externa motivationen är belöningar och straff för det man åstadkommer, det kan innebära att man utför arbetet för att nå en befordran eller för att undvika att få sparken. En intern motivation handlar istället om att man gör något av egen vilja. Det är inte externa faktorer som gör att personen i fråga vill utföra uppgiften utan för att den känns relevant och viktig.

\begin{quote}
\[...\] Sen kan ju jag som chef säga riktningen och säga ``Vi ska ditåt, vi ska uppför det där berget.''. I min värld säger de ``Bra, stå åt sidan så fixar vi det.'' istället för ``Jaha, hur ska vi göra det?''.
\end{quote}

John McCarthy berättar om hur han ser på det moderna ledarskapet. Han berättar om hur en ledare inte ska veta hur något göras, det bör lämnas till experter inom området som jobbar med exempelvis koden varje dag. Istället är det upp till ledaren att  visa vilken riktning man ska ta och varför.

Detta är något som orsakade orolighet när Ericsson valde att gå över till ett agilt arbetssätt för ungefär sex år sedan. John McCarthy beskriver hur han kunde känna att chefsrollen blev något redundant. Ibland kunde frustration uppstå när chefen inte längre stod för lösningen. Vidare konstaterar han dock att när han tidigare var tvungen att visa vägen blev det väldigt ofta fel.

Det agila arbetssättet har även medfört positiva förändringar som att det blivit mycket roligare att vara chef jämfört med tidigare.