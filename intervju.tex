%% Intervju
\section{Intervju}
\label{sec:intervju}

Innan McCarthy formellt sett hamnade på en chefsposition så började han leda människor genom att han tog på sig en ledarroll och försökate få folk att följa honom i det dagliga arbetet. Efter det började han få mer förtroende genom att få gruppledarrollen för att till slut bli chef. John tror att han till slut hamnade på en ledarroll för att han är väldigt driven, och att han på så vis kan få personer bakom sig. Han betonar också vikten av att våga om man vill nå en chefsposition. Det är upp till varje anställd att själv ta initiativet och börja leda. Han tycker också att det är viktigt att inte vara rädd för att misslyckas på sin resa mot en chefsrol.

När det gäller egenskaper en god ledare ska ha nämns \textit{förmågan att få folk bakom sig} flera gånger av John McCarthy. Det finns en signifikant skillnad mellan denna egenskap och \textit{förmågan att delegera arbete} som kanske oftare förknippas med ledarrollen. I John McCarthys formulering ingår någon slags motivation. Det handlar inte bara om att berätta för sina anställda vad de ska göra utan också varför de ska göra det.

Under en föreläsning i kursen togs begreppen intern och extern motivation upp~\citep{motivation}. Den externa motivationen är belöningar och straff för det man åstadkommer, det kan innebära att man utför arbetet för att nå en befordran eller för att undvika att få sparken. En intern motivation handlar istället om att man gör något av egen vilja. Det är inte externa faktorer som gör att personen i fråga vill utföra uppgiften utan för att den känns relevant och viktig.

På frågan hur John McCarthy tror att hans anställda ser på honom som ledare svarar han öppen och energisk. Han har också höga förväntingar på sin personal vilken han tror sina anställda känner av. Detta är också något som han också pratar om med dem.

\begin{quote}
 Sen kan ju jag som chef säga riktningen och säga ``Vi ska ditåt, vi ska uppför det där berget.''. I min värld säger de: ``Bra, stå åt sidan så fixar vi det.'' istället för: ``Jaha, hur ska vi göra det?''.
\end{quote}

John McCarthy berättar om hur han ser på det moderna ledarskapet. Han berättar om hur en ledare inte nödvändigtvis ska veta hur något göras, det bör lämnas till experter inom området som jobbar med exempelvis koden varje dag. Istället är det upp till ledaren att  visa vilken riktning man ska ta och varför.

Detta är något som orsakade orolighet när Ericsson valde att gå över till ett agilt arbetssätt för ungefär sex år sedan. John McCarthy beskriver hur han kunde känna att chefsrollen blev mer redundant och kunde därför ibland uppstå frustration när chefen inte längre stod för alla lösningar. Vidare konstaterar han dock att när han tidigare var tvungen att visa vägen blev det väldigt ofta fel. Innan Ericsson övergång till ett agilt arbetssätt kände ofta John McCarthy att han blev en flaskhals, vilket blev till ett stort stressmoment. Han berättar att i och med övergången till ett agilt arbetsätt så har denna känsla försvunnit och han kan nu arbeta mer effektivt med att skapa förutsättningar för sina anställda att lösa problem själva. John McCarthy poängterar flera gånger att han tycker att ledarrollen är mycket roligare i Ericssons nya organisation. Samtidigt säger han också att övergången till det nya arbetssättet har varit jobbigt på många sätt. Han tycker heller inte att övergågen är helt klart. Han tror att det kommer ta ett bra tag innan kulturen är helt ändrad.

Vidare berättar John McCarthy att just synen på hur insatt i detaljfrågor en chef bör vara kan leda till kulturkrockar. På många platser runt om i världen är fortfarande tankesättet att det är upp till chefen att ha kunskap. John McCarthy anser sig dock hittat en lösning i att han är nyfiken på vad som sker. Det finns en intern motivation att lära sig om vissa delar av projektet vilket leder till att han kan svara på vissa frågor men går man in i detalj är det bättre att personen pratar med de anställda som jobbar med utvecklingen dagligen. Han betonar också vikten av att ta in rätt personer från skolan eller inom ett visst expertområde för att kunna nå en förändring.

På McCarhys avdelning på Ericsson skickas det årligen ut en enkät som de kallar dialogresultat som innehåller frågor om hur de anställda mår på jobbet, hur motiverade de är, om de känner sig utmnanade och vilken syn de har på sin chef. Detta resulterar till något McCarthy hänvisar till som ett sorts ledarskapsindex som ger svar på hur arbetsgruppen mår, hur många som känner sig motiverade, hur många som helt enkelt slutat bry sig mm. McCarthy betonar dock att det också är väldigt viktigt att han som ledare träffarna sina anställda ofta och på så sätt kan känna av känslan i en grupp för att på så sätt skapa sig en bild av hur de anställda mår.

Generellt ger John McCarthy ett ödmjukt intryck. Han är noga med att framhäva att kunskapen ska finnas hos de anställda som utför arbetet. När han får frågan om hur han ser sig själv som ledare använder han ord som transparent och energisk. Detta är också något som han tror att hans anställda ser hos honom. Dock tror han att hans anställda ser på honom som en ledare med höga krav. Ambitionen är att ta in folk som är bättre än han själv.