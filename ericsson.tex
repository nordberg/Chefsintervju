% Om Ericsson
\subsection{Ericsson}
Ericssons bygger på ett pyramidsystem med VD:n Hans Vestberg i toppen. John McCarthy poängterar dock att pyramiden inte innebär att anställda ska se honom som någon slags gud som står på ett berg och pekar. Istället ska man visa vägen, sätta en riktning och sedan se till att kunniga personer finns som kan nå dit också.

Som tidigare nämnt jobbar Ericsson just nu på en stor kulturändring i sin övergång till det agila arbetssättet. John McCarthy nämner att just beslutsfattning och hur besluten når ut i företaget är något som behöver jobbas på.

\begin{quote}
Vi behöver jobba med [beslutsfattning]. Jag känner att de beslut som vi tar inte ripplar ner i organisationen om man säger så. [...] Vi kan vara jättesams i ledarteamet men sen när man kommer ut och ska propagera sitt beslut kan jag få frågan ``Hur har ni tänkt här?''
\end{quote}

När en strategi ska tas fram görs detta i samråd med parallella organisationer utanför Stockholm (Både svenska och utländska). Att sedan förklara varför denna strategi ska sättas är något John McCarthy lägger stor vikt vid. Han menar på att en utvecklare som ser sammanhanget och förstår varför just den komponent han eller hon utvecklar är viktig för den större bilden kommer göra ett bättre jobb. Det är dock något som, enligt han själv, är en av de svårare uppgifterna han har i sin roll.