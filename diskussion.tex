%% Diskussion
\section{Diskussion}

I detta avsnitt jämförs kursmaterialet med intervjusvaren.

\subsection{Vad är ledarskap?}
När det gäller egenskaper en god ledare ska ha nämns \textit{förmågan att få folk bakom sig} flera gånger av John McCarthy. Det finns en signifikant skillnad mellan denna egenskap och \textit{förmågan att delegera arbete} som kanske oftare förknippas med ledarrollen. I John McCarthys formulering ingår någon slags motivation. Det handlar inte bara om att berätta för sina anställda vad de ska göra utan också varför de ska göra det.

I kursmaterialet~\citep{leadership} definieras ledarskap med ett citat från Gary Yukl~\citep{yukl} som beskriver det John McCarthy sa:
\begin{quote}
Leadership is the process of influencing others to understand 
and agree about what needs to be done and how it can be 
done effectively, and the process of facilitating individual and 
collective efforts to accomplich the shared objectives.
\end{quote}

Kurslitteraturens beskrivning av ledarskap stämmer således med den bild som John McCarthy gav. Kursmaterialet nämner \textit{Expectancy theory} som ett exempel på en motivationsmodell~\cite{motivation} i kunskapsarbete. Att kunna motivera är en viktigt egenskap för en ledare och John McCarthy berättade under avsnitt \ref{sec:intervju} att han tror att de han jobbar med vet vad som förväntas av dem.

Framtidens ledarskap kan komma att se annorlunda ut beroende på hur organisationen förändras. John McCarthy håller sig därför uppdaterad med modern litteratur men har även läst över trettio år gamla teorier som håller än idag. Det krävs en balans för att lyckas precis som med organisationen; om organisationen ständigt nyanställer och förnyar sig bildas ingen stadig grund och om organisationen aldrig förnyas halkar den efter i takt med att samhället utvecklas.

\subsection{Agil projektledning}
Ett agilt arbetssätt som utgår ifrån det agila manifestet som omnämns i kursmaterialet~\citep{projekt} är något som John McCarthy och resten av Ericsson strävar efter. John McCarrthy berättar under avsnitt \ref{sec:bakgrund} om den kulturförändring som pågår och svårigheterna som uppstår. Med den något omoderna pyramidstrukturen är arbetsdelegationen enkel när instruktionerna uppifrån är tydliga. Nackdelen är att alla (generaliserat) vänder sig uppåt med frågor vilket skapar en flaskhals i systemet när oväntade komplikationer uppstår. Med en agil projektledning är det viktigt att målen är tydliga eftersom att hierarkin plattas ut och medarbetarna interagerar med varandra för att lösa problemen istället för att vända sig uppåt. Det är alltså ledningen som definierar målen men det är upp till folket att bestämma riktningen.

En av punkterna i det agila manifestet lyder
\begin{quote}
Det är viktigare hur man hanterar förändring än att man följer en förutbestämd plan.
\end{quote}
Innan kulturövergången på Ericsson var det projektledarna som satte riktningen och det blev oftast fel. Det är därför viktigt att riktningen kan ändras under projektets gång för att det slutgiltiga målet ska nås. Nyckeln till riktningsändringen är interaktion inom projektgruppen för att så tidigt som möjligt veta om man gör rätt eller fel. Kursmaterialet säger att det är viktigt att sätta upp kortsiktiga delmål som är kontrollerbara vid målformuleringen~\citep{projekt} men dessa mål kan komma att ändras under projektets gång.

\subsection{Ericsson, ett kunskapsföretag}
I kursmaterialet~\citep{teknik} definieras ett kunskapsföretag som en organisation med hög utbildningsnivå, låg styrning, flexibilitet, ej inarbetade lösningar, nära och förtroendebaserat, subjektivt och svårt samt hög specialisering. Den bild John McCarthy har gett oss stämmer helt överens med dessa punkter. Att vara ledare inom en sådan organisation är mycket svårt men samtidigt mycket givande. De svåra besluten och det ibland ickeexisterande kunskapsområdet gör ledarrollen mycket utmanande.