%% Diskussion
\section{Diskussion}

\subsection{Vad är ledarskap?}
Som tidigare nämnts poängterade John McCarthy att ledarskap handlar om att få folk bakom sig och motivera dem till att nå målen. I kursmaterialet~\citep{leadership} definieras ledarskap med ett citat från Gary Yukl~\citep{yukl} som beskriver det John McCarthy sa:
\begin{quote}
Leadership is the process of influencing others to understand 
and agree about what needs to be done and how it can be 
done effectively, and the process of facilitating individual and 
collective efforts to accomplich the shared objectives.
\end{quote}

Kurslitteraturens beskrivning av ledarskap stämmer således med den bild som John McCarthy gav.

\subsection{Agil projektledning}
Ett agilt arbetssätt som utgår ifrån det agila manifestet som omnämns i kursmaterialet~\citep{projekt} är något som John McCarthy och resten av Ericsson strävar efter. John McCarrthy berättar under avsnitt \ref{sec:bakgrund} om den kulturförändring som pågår och svårigheterna som uppstår. Med den något omoderna pyramidstrukturen är arbetsdelegationen enkel när instruktionerna uppifrån är tydliga. Nackdelen är att alla (generaliserat) vänder sig uppåt med frågor vilket skapar en flaskhals i systemet när oväntade komplikationer uppstår. Med en agil projektledning är det viktigt att målen är tydliga eftersom att hierarkin plattas ut och medarbetarna löser problemen istället för att vända sig uppåt.
